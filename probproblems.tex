\section{Теория вероятностей: Задачи}
\begin{problem}{(ФКН ВШЭ: Теория вероятностей: Листок 3)}
	Пусть $ \xi $ некоторая случайная величина. При каком $ a \in \mathbb{R} $ достигается минимальное значение $ f(a) =\mathbb{E}\left[(\xi-a)^{2}\right] $
\end{problem}
\begin{solution}
	Раскроем по линейности матожидание. $ \mathbb{E}\left[(\xi-a)^{2}\right]=\mathbb{E}\left[\xi^{2}\right]-2 a \mathbb{E}[\xi]+a^{2} $.\\
	Прибавим и отнимем $ (\mathbb{E}(\xi))^2 $\\
	
	$f(a)=(a-\mathbb{E}[\xi])^{2}+\mathbb{E}\left[\xi^{2}\right]-(\mathbb{E}[\xi])^{2}=(a-\mathbb{E}[\xi])^{2}+\mathbb{D}[\xi]$
	Так как дисперисия не зависит от параметра $ a $ $ \Rightarrow $ минимум достигается при $ (a-\mathbb{E}[\xi])^{2} = 0 $ т.е $ a =\mathbb{E}[\xi] $
\end{solution}
\begin{problem}{(ФКН ВШЭ: Теория вероятностей: Листок 3)}
	Вычислить $\mathbb{E}[\xi], \mathbb{D}[\xi] u \mathbb{E}\left[3^{\xi}\right]$, если $ \xi $ - это
	\begin{enumerate}
		\item пуассоновская случайная величина с параметром $ \lambda \ge 0 $
		
		\item геометрическая случайная величина с параметром $ p \in (0,1) $
	\end{enumerate}	
\end{problem}

\begin{solution}
	\begin{enumerate}
		\begin{enumerate}
			\item
			
			
			$\mathbb{P}[\xi=k]=\frac{\lambda^{k}}{k !} e^{-\lambda}, k \in \mathbb{N}$ \\
			$\mathbb{E}[\xi]=\sum_{k=1}^{\infty} k \cdot \frac{\lambda^{k}}{k !} e^{-\lambda}=\frac{\lambda}{e^{\lambda}} \sum_{k=1}^{\infty} \frac{\lambda^{k-1}}{(k-1) !}=\frac{\lambda}{e^{\lambda}} \cdot e^{\lambda}=\lambda$ \\
			\item
			$\mathbb{D}[\xi]=\mathbb{E}\left[\xi^{2}\right]-(\mathbb{E}[\xi])^{2}=\sum_{k=1}^{\infty} k^{2} \cdot \frac{\lambda^{k}}{k !} e^{-\lambda}-\lambda^{2}=\frac{1}{e^{\lambda}} \sum_{k=1}^{\infty} \frac{k \lambda^{k}}{(k-1) !}-\lambda^{2}$
			
			$\frac{1}{e^{\lambda}} \sum_{k=0}^{\infty} \frac{(k+1) \lambda^{k+1}}{k !}-\lambda^{2}=\frac{\lambda^{2}}{e^{\lambda}} \sum_{k=1}^{\infty} \frac{\lambda^{k-1}}{(k-1) !}+\frac{\lambda}{e^{\lambda}} \underbrace{\sum_{k=0}^{\infty} \frac{\lambda^{k}}{k !}}_{=e^{\lambda}}-\lambda^{2}=\lambda$
			
			\item
			$\mathbb{E}\left[3^{\xi}\right]=\sum_{k=1}^{\infty} 3^{k} \cdot \frac{\lambda^{k}}{k !} e^{-\lambda}=e^{-\lambda} \underbrace{\sum_{k=1}^{\infty} \frac{(3 \lambda)^{k}}{k !}}_{=e^{3 \lambda}}=e^{2 \lambda}$
		\end{enumerate}
	
	\begin{enumerate}
		\item 
		$\mathbb{P}[\xi=k]=p(1-p)^{k-1}, k \in \mathbb{N}$
		$\mathbb{E}[\xi]=\sum_{k=1}^{\infty} k \cdot(1-p)^{k-1} p=p \sum_{k=1}^{\infty} k(1-p)^{k-1}$
		
		$$
		\sum_{k=1}^{\infty} q^{k}=\frac{q}{1-q} \quad q \in(0,1)
		$$
		
		$$
		\frac{\partial}{\partial q} \sum_{k=1}^{\infty} q^{k}=\sum_{k=1}^{\infty} k q^{k-1}
		$$
		
		$q=1-p$
		
		$$
		\mathbb{E}[\xi]=p\left(\frac{\partial}{\partial q}\left(\frac{q}{1-q}\right) | q=1-p\right)=p\left(\frac{1-q+q}{(1-q)^{2}} | q=1-p\right)=p \cdot \frac{1}{p^{2}}=\frac{1}{p}
		$$
		
		\item 
		$$
		\mathbb{D}[\xi]=\mathbb{E}\left[\xi^{2}\right]-(\mathbb{E}[\xi])^{2}=\mathbb{E}\left[\xi^{2}-\xi+\xi\right]-(\mathbb{E}[\xi])^{2}=\mathbb{E}[\xi(\xi-1)]+\mathbb{E}[\xi]-(\mathbb{E}[\xi])^{2}
		$$
		
		$$
		\begin{aligned} \mathbb{E}[\xi(\xi-1)] = p \sum_{k=1}^{\infty} k(k-1) q^{k-1}=p \cdot \frac{\partial}{\partial q}\left(\sum_{k=1}^{\infty}(k-1) q^{k}\right)=p \cdot \frac{\partial}{\partial q}\left(q^{2} \sum_{k=2}^{\infty}(k-1) q^{k-2}\right)=\\=p \cdot \frac{\partial}{\partial q}\left(q^{2} \cdot \frac{\partial}{\partial q} \sum_{k=2}^{\infty} q^{k-1}\right)=\\\left.=p \cdot \frac{\partial}{\partial q}\left(q^{2} \cdot \frac{\partial}{\partial q} \sum_{k=1}^{\infty} q^{k}\right)=p \cdot \frac{\partial}{\partial q}\left(q^{2} \cdot \frac{\partial}{\partial q}\right)\right)=p \cdot \frac{\partial}{\partial q}\left(\frac{q^{2}}{(1-q)^{2}}\right)=p\left(\frac{-2 q}{(q-1)^{3}}\right)=\\=\{p=1-q\}=p\left(\frac{-2+2 p}{-p^{3}}\right)=\frac{2(p-1)}{-p^{2}}=\frac{2(1-p)}{p^{2}} \end{aligned}
		$$
		
		$$\mathbb{D}[\xi]=\mathbb{E}[\xi(\xi-1)]+\mathbb{E}[\xi]-(\mathbb{E}[\xi])^{2}=\frac{2(1-p)}{p^{2}}+\frac{1}{p}-\frac{1}{p^{2}}=\frac{2-2 p+p-1}{p^{2}}=\frac{1-p}{p^{2}}
		$$
		
		\item 
	
		$$\mathbb{E}\left[3^{\xi}\right]=\sum_{k=1}^{\infty} 3^{k} \cdot p(1-p)^{k-1}=3 p \sum_{k=1}^{\infty} 3^{k-1} \cdot(1-p)^{k-1}$$
		
		Для сходимости необходимо, чтобы $3(1-p)<1$
		
		$$
		\mathbb{E}\left[3^{\xi}\right]=\frac{3 p}{1-3(1-p)}
		$$
	\end{enumerate}

	\end{enumerate}


\end{solution}


\begin{problem}(ФКН ВШЭ: Теория Вероятностей: Листок 3)
	Множество из $ k $ шаров случайно раскладывают по $ m $ ящикам. Случайная величина $ \xi $ равна количеству пустых ящиков при таком случайном размещении. Найдите $ \mathbb{E}\xi \text{ и }\mathbb{D}\xi$
	, если (а) шары неразличимы, (b) различимы.
\end{problem}

\begin{solution}
	\begin{enumerate}
		\item 
		Для начала посчитаем сколько случайных размещений $ k $ шаров по $ m $ ящикам.
		Пусть ящики стоят в ряд и мы раскладываем по ним шары. $ m $ ящикам соответствуют $ m-1 $ перегородка.(Две перегородки по краям никак не двигаются). Перемешав шары и внутренние перегородки (шары неразличимы) получим $ \displaystyle 
		C_{k+m-1}^{k}$ - различных способой расставить шары по ящикам.
		 Теперь выберем $ \xi = i $ ящиков которые будут пустыми.
		 Это можно сделать $ \displaystyle C^{i}_{m} $ способами.
		 У нас осталось $ m - i  $ непустых ящиков. Это приводит нас к такому уравнению
		 $$
		 y_{1}+\cdots+y_{m-i}=k
		 $$
		 ,где все $ \forall j \quad y_{j} \ge 0 $
		  Вычтем из каждого $ y_{j} $ единицу, что соответвует тому, что никакой из ящиков не может опустеть в результате случайного расскладывания шаров по ящикам. Получаем уравнение
		  $$ 
		  x_{1}+\cdots+x_{m-i} = k - (m - i)
		  $$
		  Количество целочисленных решений этого уравнения равно расстановке $ k - (m -i) $ шаров по $ m - i $  ящику.
		  Эта задача уже нами решена.
		  
		  $$ C^{(m -i)}_{k - (m -i) + ((m -i)-1) } = C^{m-i}_{k-1}$$
		  
		  Отсюда получаем $$ P(\xi = i) = \frac{C_{m}^{i} C^{m-i}_{k-1}}{C_{k+m-1}^{k}}$$
		  
		  $$
		  \mathbb{E} \xi = \sum_{i=0}^{m} i \cdot \frac{C_{m}^{i} C^{m-1-i}_{k-1}}{C_{k+m-1}^{m-1}} 
		  $$
		  $
		  P(\xi=i)
		  $ Ни что иное как гипергеометрическая функция с параметрами $ N = k + m -1 $, $ D = m $ , $ l = i , n = m -1  $
		  
	$$
	f(k ; N, D, n) = 
	\frac{C_{D}^{l} C_{N-D}^{n-l}}{C_{N}^{n}}
	$$
	
	Выведем матожидание для гипергеометрической функции
	
	$$ 
		l \cdot C^{l}_{D} = \frac{D !}{(l-1) !(D-l) !}=\frac{D(D-1) !}{(l-1) !((D-1)-(l-1)) !} = D C^{l-1}_{D-1}
	$$
	
	$$
		l f(l ; N, D, n) =  \frac{D C^{l-1}_{D-1} C_{(N-1)-(D-1)}^{(n-1)-(l-1)}\cdot}{\frac{N}{n} C^{n-1}_{N-1}}
	$$
	
	$$	 
	\mathbb{E} \xi = \frac{Dn}{N} \sum_{l} 
	\frac{ C^{l-1}_{D-1} C_{(N-1)-(D-1)}^{(n-1)-(l-1)}\cdot}{C^{n-1}_{N-1}}
	$$	
	
	Выражение под знаком суммы  равно 1
	
	$$ \mathbb{E} \xi = \frac{Dn}{N}=\frac{m(m-1)}{(k+m-1)} $$
	 
	\end{enumerate}
\end{solution}	