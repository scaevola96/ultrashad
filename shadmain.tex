\documentclass[a4paper,12pt]{article}

%%% Работа с русским языком
\usepackage{cmap}					% поиск в PDF
\usepackage[T2A]{fontenc}			% кодировка
\usepackage[utf8]{inputenc}			% кодировка исходного текста
\usepackage[english,russian]{babel}	% локализация и переносы
%%
\usepackage{cmap} 
\usepackage{ gensymb }
\usepackage[unicode]{hyperref}
\usepackage{ textcomp }
\usepackage{datetime}
\usepackage{physics}
\usepackage{cancel}
\usepackage{mathtools}
\usepackage[margin=0.7in]{geometry}
\usepackage{fancyhdr}
\pagestyle{fancy}
%%

%%% Дополнительная работа с математикой
\usepackage{amsfonts,amssymb,amsthm,mathtools} % AMS
\usepackage{amsmath}
\usepackage{icomma} % "Умная" запятая: $0,2$ --- число, $0, 2$ --- перечисление

%% Теоремы
\newtheorem{exercise}{Упражнение}[section]

\theoremstyle{definition}
\newtheorem{definition}{Определение}
\theoremstyle{plain}
\newtheorem{theorem}{Теорема}
\newtheorem{lemma}{Лемма}
\newtheorem{example}{Пример}
\newtheorem{problem}{Задача}
\newtheorem*{solution}{Решение}

%
\theoremstyle{plain}
%\newtheorem{proof}{Доказательство}
%
\newtheorem{statement}{Утверждение}

\theoremstyle{remark}
\newtheorem*{properties}{Свойства}

%%

\fancyhead[L]{\footnotesize ШАД Яндекс}
\fancyhead[RO]{Последняя компиляция: \today }
\fancyfoot[R]{\thepage}
\fancyfoot[C]{}




%% Номера формул
%\mathtoolsset{showonlyrefs=true} % Показывать номера только у тех формул, на которые есть \eqref{} в тексте.

%% Шрифты
\usepackage{euscript}	 % Шрифт Евклид
\usepackage{mathrsfs} % Красивый матшрифт

%% Свои команды
\DeclareMathOperator{\sgn}{\mathop{sgn}}

%% Перенос знаков в формулах (по Львовскому)
\newcommand*{\hm}[1]{#1\nobreak\discretionary{}
	{\hbox{$\mathsurround=0pt #1$}}{}}

%%% Работа с картинками
\usepackage{graphicx}  % Для вставки рисунков
\graphicspath{{images/}{images2/}}  % папки с картинками
\setlength\fboxsep{3pt} % Отступ рамки \fbox{} от рисунка
\setlength\fboxrule{1pt} % Толщина линий рамки \fbox{}
\usepackage{wrapfig} % Обтекание рисунков и таблиц текстом

%%% Работа с таблицами
\usepackage{array,tabularx,tabulary,booktabs} % Дополнительная работа с таблицами
\usepackage{longtable}  % Длинные таблицы
\usepackage{multirow} % Слияние строк в таблице


%%% Заголовок
\title{Школа анализа данных}
\date{\today}


\begin{document} % конец преамбулы, начало документа
	
	\begin{definition}
	Пусть $R$ — произвольное кольцо. Если существует такое целое положительное число $n$, что $\forall r\in R$ выполняется равенство
	
	$$
	n\cdot r=\underbrace{r+\cdots +r}_{n} = 0
	$$
	то наименьшее из таких чисел n называвается характеристикой поля и обозначается $ {\mathop  {{\mathrm  {char}}}}R$
	Если такого числа не существует то $ {\mathop  {{\mathrm  {char}}}}R = 0$
\end{definition}

\begin{theorem}[\protect{\cite[ВА II \S 3.4 ]{kostrikin2}}]
	\label{thm:thm1}
	\mylabel{text:thm1}{Критерий диагонализируемости}
	Линейный оператор $ A $ с простым спектром диагонализируем
\end{theorem}

\begin{theorem}[\protect{\cite[ВА II \S 3.4 ]{kostrikin2}}]
	\label{thm:thm2}
	\mylabel{text:thm2}{Критерий диагонализируемости}
	Пусть $ A $ линейный оператор на конечномерном векторном пространстве $ V $ на полем $\mathfrak{K}$. Для диагонализируемости $ \mathcal{A} $ необходимо и достаточно выполнение двух условий:
	\begin{itemize}
		\item все корни характеристического многочлена $\chi_{\mathcal{A}}(t)$ лежат в $ \mathfrak{K} $
		\item 
		геометрическая кратность каждого собственного значения  $ \lambda $ совпадает с его алгебраической кратностью.  
	\end{itemize}
\end{theorem}
	\section{Комбинаторика}
\begin{lemma}[Полиномиальная формула]
	$$\left(x_{1}+x_{2}+\ldots+x_{m}\right)^{n} = \left(x_{1}+x_{2}+\ldots+x_{m}\right)\cdot\left(x_{1}+x_{2}+\ldots+x_{m}\right) \ldots \left(x_{1}+x_{2}+\ldots+x_{m}\right)$$
	
	$$
	k_{1}+k_{2}+\ldots+k_{m}=n
	$$
\end{lemma}

	\section{Линейная Алгебра: Задачи}
\begin{problem}
	Существуют ли матрицы $A$ b $ B $ такие что $A B-B A=I$
\end{problem}

\begin{solution}
	Нет. $\operatorname{tr}(A B)=\operatorname{tr}(B A)$  $\operatorname{tr}(A B-B A)=0$ тогда как
	$\operatorname{tr}(I) \neq 0$
\end{solution}

\begin{problem}[ШАД Экзамен 25 мая 2019]
 Матрицы $ A $ и $ B $ таковы, что $ A^2 = A $ и $ B^2 = B $ и матрица $ E - (A + B) $ обратима. Докажите, что $ \operatorname{rk} A = \operatorname{rk} B $

\end{problem}
\begin{solution}
	$ C = E - (A+B) $. Домножим на $ A C =  A \left(E - (A+B)\right) = A - (A^2 + AB) = - AB $. $ CB = \left(E - (A+B)\right) B  = - AB $, Получаем $ AC = CB \rightarrow A = CBC^{-1}$
	
\end{solution}	

\begin{solution}
$\operatorname{rk}(A+B) \leqslant \operatorname{rk}A+\operatorname{rk}B$

$$
\left\{\begin{array}{l}{\operatorname{rk} (E-(A+B))} \leqslant \operatorname{rk}(E-A) + \operatorname{rk} B \\ {\operatorname{rk}(E-(A+B))}  \leqslant \operatorname{rk}(E-B) + \operatorname{rk} A \end{array}\right.
$$


$$
\operatorname{rk}(A) = n_1, \quad \operatorname{rk}(B) = n_2
$$

По свойству проекционных операторов 
$
 \operatorname{rk}(E-A) = n - n_1 , \operatorname{rk}(E-B) = n - n_2
$

$$
\left\{\begin{array}{l}
{n\leqslant n - n_1 + n_2}\\{n\leqslant n - n_2 + n_1}
		
\end{array}\right. \Longrightarrow n_1 = n_2
$$		
\end{solution}

\begin{problem}[ШАД Экзамен 1 июня 2019]
	При каком значении параметра $ a \in \mathbb{R}  $ матрицы
	$$
	A=\left(\begin{array}{cc}{1} & {4-a-a^{2}} \\ {2} & {-1}\end{array}\right) \quad B=\left(\begin{array}{cc}{-a-1} & {3} \\ {3} & {-5}\end{array}\right)
	$$
	могут быть матрицами одной и той же билинейной формы $
	V \times V \rightarrow \mathbb{R}
	$ в различных базисах
	
\end{problem}

\begin{solution}
	$$
	A=C^{T} B C
	\Longrightarrow 
	\operatorname{r k} A=\operatorname{r k} B
	$$
	
	$$
	\left\{\begin{array}{l}{\operatorname{det} A \neq 0} \\ {\operatorname{det} B \neq 0}\end{array}\right.
	$$
	
	$$
	\begin{array}{l}{a \neq \frac{4}{5}} \\ {a \neq \frac{1}{2}(-1-\sqrt{19})} \\ {a \neq \frac{1}{2}(-1+\sqrt{19})}\end{array}
	$$
	
	Заметим, что  $
	B^{T}=B \Longrightarrow 
	A^{T}=A , 
	4-a-a^{2}=2 , a = 1 , a = -2	
	$
\end{solution}


				
	\section{Теория вероятностей: Задачи}
\begin{problem}{(ФКН ВШЭ: Теория вероятностей: Листок 3)}
	Пусть $ \xi $ некоторая случайная величина. При каком $ a \in \mathbb{R} $ достигается минимальное значение $ f(a) =\mathbb{E}\left[(\xi-a)^{2}\right] $
\end{problem}
\begin{solution}
	Раскроем по линейности матожидание. $ \mathbb{E}\left[(\xi-a)^{2}\right]=\mathbb{E}\left[\xi^{2}\right]-2 a \mathbb{E}[\xi]+a^{2} $.\\
	Прибавим и отнимем $ (\mathbb{E}(\xi))^2 $\\
	
	$f(a)=(a-\mathbb{E}[\xi])^{2}+\mathbb{E}\left[\xi^{2}\right]-(\mathbb{E}[\xi])^{2}=(a-\mathbb{E}[\xi])^{2}+\mathbb{D}[\xi]$
	Так как дисперисия не зависит от параметра $ a $ $ \Rightarrow $ минимум достигается при $ (a-\mathbb{E}[\xi])^{2} = 0 $ т.е $ a =\mathbb{E}[\xi] $
\end{solution}
\begin{problem}{(ФКН ВШЭ: Теория вероятностей: Листок 3)}
	Вычислить $\mathbb{E}[\xi], \mathbb{D}[\xi] u \mathbb{E}\left[3^{\xi}\right]$, если $ \xi $ - это
	\begin{enumerate}
		\item пуассоновская случайная величина с параметром $ \lambda \ge 0 $
		
		\item геометрическая случайная величина с параметром $ p \in (0,1) $
	\end{enumerate}	
\end{problem}

\begin{solution}
	\begin{enumerate}
		\begin{enumerate}
			\item
			
			
			$\mathbb{P}[\xi=k]=\frac{\lambda^{k}}{k !} e^{-\lambda}, k \in \mathbb{N}$ \\
			$\mathbb{E}[\xi]=\sum_{k=1}^{\infty} k \cdot \frac{\lambda^{k}}{k !} e^{-\lambda}=\frac{\lambda}{e^{\lambda}} \sum_{k=1}^{\infty} \frac{\lambda^{k-1}}{(k-1) !}=\frac{\lambda}{e^{\lambda}} \cdot e^{\lambda}=\lambda$ \\
			\item
			$\mathbb{D}[\xi]=\mathbb{E}\left[\xi^{2}\right]-(\mathbb{E}[\xi])^{2}=\sum_{k=1}^{\infty} k^{2} \cdot \frac{\lambda^{k}}{k !} e^{-\lambda}-\lambda^{2}=\frac{1}{e^{\lambda}} \sum_{k=1}^{\infty} \frac{k \lambda^{k}}{(k-1) !}-\lambda^{2}$
			
			$\frac{1}{e^{\lambda}} \sum_{k=0}^{\infty} \frac{(k+1) \lambda^{k+1}}{k !}-\lambda^{2}=\frac{\lambda^{2}}{e^{\lambda}} \sum_{k=1}^{\infty} \frac{\lambda^{k-1}}{(k-1) !}+\frac{\lambda}{e^{\lambda}} \underbrace{\sum_{k=0}^{\infty} \frac{\lambda^{k}}{k !}}_{=e^{\lambda}}-\lambda^{2}=\lambda$
			
			\item
			$\mathbb{E}\left[3^{\xi}\right]=\sum_{k=1}^{\infty} 3^{k} \cdot \frac{\lambda^{k}}{k !} e^{-\lambda}=e^{-\lambda} \underbrace{\sum_{k=1}^{\infty} \frac{(3 \lambda)^{k}}{k !}}_{=e^{3 \lambda}}=e^{2 \lambda}$
		\end{enumerate}
	
	\begin{enumerate}
		\item 
		$\mathbb{P}[\xi=k]=p(1-p)^{k-1}, k \in \mathbb{N}$
		$\mathbb{E}[\xi]=\sum_{k=1}^{\infty} k \cdot(1-p)^{k-1} p=p \sum_{k=1}^{\infty} k(1-p)^{k-1}$
		
		$$
		\sum_{k=1}^{\infty} q^{k}=\frac{q}{1-q} \quad q \in(0,1)
		$$
		
		$$
		\frac{\partial}{\partial q} \sum_{k=1}^{\infty} q^{k}=\sum_{k=1}^{\infty} k q^{k-1}
		$$
		
		$q=1-p$
		
		$$
		\mathbb{E}[\xi]=p\left(\frac{\partial}{\partial q}\left(\frac{q}{1-q}\right) | q=1-p\right)=p\left(\frac{1-q+q}{(1-q)^{2}} | q=1-p\right)=p \cdot \frac{1}{p^{2}}=\frac{1}{p}
		$$
		
		\item 
		$$
		\mathbb{D}[\xi]=\mathbb{E}\left[\xi^{2}\right]-(\mathbb{E}[\xi])^{2}=\mathbb{E}\left[\xi^{2}-\xi+\xi\right]-(\mathbb{E}[\xi])^{2}=\mathbb{E}[\xi(\xi-1)]+\mathbb{E}[\xi]-(\mathbb{E}[\xi])^{2}
		$$
		
		$$
		\begin{aligned} \mathbb{E}[\xi(\xi-1)] = p \sum_{k=1}^{\infty} k(k-1) q^{k-1}=p \cdot \frac{\partial}{\partial q}\left(\sum_{k=1}^{\infty}(k-1) q^{k}\right)=p \cdot \frac{\partial}{\partial q}\left(q^{2} \sum_{k=2}^{\infty}(k-1) q^{k-2}\right)=\\=p \cdot \frac{\partial}{\partial q}\left(q^{2} \cdot \frac{\partial}{\partial q} \sum_{k=2}^{\infty} q^{k-1}\right)=\\\left.=p \cdot \frac{\partial}{\partial q}\left(q^{2} \cdot \frac{\partial}{\partial q} \sum_{k=1}^{\infty} q^{k}\right)=p \cdot \frac{\partial}{\partial q}\left(q^{2} \cdot \frac{\partial}{\partial q}\right)\right)=p \cdot \frac{\partial}{\partial q}\left(\frac{q^{2}}{(1-q)^{2}}\right)=p\left(\frac{-2 q}{(q-1)^{3}}\right)=\\=\{p=1-q\}=p\left(\frac{-2+2 p}{-p^{3}}\right)=\frac{2(p-1)}{-p^{2}}=\frac{2(1-p)}{p^{2}} \end{aligned}
		$$
		
		$$\mathbb{D}[\xi]=\mathbb{E}[\xi(\xi-1)]+\mathbb{E}[\xi]-(\mathbb{E}[\xi])^{2}=\frac{2(1-p)}{p^{2}}+\frac{1}{p}-\frac{1}{p^{2}}=\frac{2-2 p+p-1}{p^{2}}=\frac{1-p}{p^{2}}
		$$
		
		\item 
	
		$$\mathbb{E}\left[3^{\xi}\right]=\sum_{k=1}^{\infty} 3^{k} \cdot p(1-p)^{k-1}=3 p \sum_{k=1}^{\infty} 3^{k-1} \cdot(1-p)^{k-1}$$
		
		Для сходимости необходимо, чтобы $3(1-p)<1$
		
		$$
		\mathbb{E}\left[3^{\xi}\right]=\frac{3 p}{1-3(1-p)}
		$$
	\end{enumerate}

	\end{enumerate}


\end{solution}


\begin{problem}(ФКН ВШЭ: Теория Вероятностей: Листок 3)
	Множество из $ k $ шаров случайно раскладывают по $ m $ ящикам. Случайная величина $ \xi $ равна количеству пустых ящиков при таком случайном размещении. Найдите $ \mathbb{E}\xi \text{ и }\mathbb{D}\xi$
	, если (а) шары неразличимы, (b) различимы.
\end{problem}

\begin{solution}
	\begin{enumerate}
		\item 
		Для начала посчитаем сколько случайных размещений $ k $ шаров по $ m $ ящикам.
		Пусть ящики стоят в ряд и мы раскладываем по ним шары. $ m $ ящикам соответствуют $ m-1 $ перегородка.(Две перегородки по краям никак не двигаются). Перемешав шары и внутренние перегородки (шары неразличимы) получим $ \displaystyle 
		C_{k+m-1}^{k}$ - различных способой расставить шары по ящикам.
		 Теперь выберем $ \xi = i $ ящиков которые будут пустыми.
		 Это можно сделать $ \displaystyle C^{i}_{m} $ способами.
		 У нас осталось $ m - i  $ непустых ящиков. Это приводит нас к такому уравнению
		 $$
		 y_{1}+\cdots+y_{m-i}=k
		 $$
		 ,где все $ \forall j \quad y_{j} \ge 0 $
		  Вычтем из каждого $ y_{j} $ единицу, что соответвует тому, что никакой из ящиков не может опустеть в результате случайного расскладывания шаров по ящикам. Получаем уравнение
		  $$ 
		  x_{1}+\cdots+x_{m-i} = k - (m - i)
		  $$
		  Количество целочисленных решений этого уравнения равно расстановке $ k - (m -i) $ шаров по $ m - i $  ящику.
		  Эта задача уже нами решена.
		  
		  $$ C^{(m -i)}_{k - (m -i) + ((m -i)-1) } = C^{m-i}_{k-1}$$
		  
		  Отсюда получаем $$ P(\xi = i) = \frac{C_{m}^{i} C^{m-i}_{k-1}}{C_{k+m-1}^{k}}$$
		  
		  $$
		  \mathbb{E} \xi = \sum_{i=0}^{m} i \cdot \frac{C_{m}^{i} C^{m-1-i}_{k-1}}{C_{k+m-1}^{m-1}} 
		  $$
		  $
		  P(\xi=i)
		  $ Ни что иное как гипергеометрическая функция с параметрами $ N = k + m -1 $, $ D = m $ , $ l = i , n = m -1  $
		  
	$$
	f(k ; N, D, n) = 
	\frac{C_{D}^{l} C_{N-D}^{n-l}}{C_{N}^{n}}
	$$
	
	Выведем матожидание для гипергеометрической функции
	
	$$ 
		l \cdot C^{l}_{D} = \frac{D !}{(l-1) !(D-l) !}=\frac{D(D-1) !}{(l-1) !((D-1)-(l-1)) !} = D C^{l-1}_{D-1}
	$$
	
	$$
		l f(l ; N, D, n) =  \frac{D C^{l-1}_{D-1} C_{(N-1)-(D-1)}^{(n-1)-(l-1)}\cdot}{\frac{N}{n} C^{n-1}_{N-1}}
	$$
	
	$$	 
	\mathbb{E} \xi = \frac{Dn}{N} \sum_{l} 
	\frac{ C^{l-1}_{D-1} C_{(N-1)-(D-1)}^{(n-1)-(l-1)}\cdot}{C^{n-1}_{N-1}}
	$$	
	
	Выражение под знаком суммы  равно 1
	
	$$ \mathbb{E} \xi = \frac{Dn}{N}=\frac{m(m-1)}{(k+m-1)} $$
	 
	\end{enumerate}
\end{solution}	
	\section{Комбинаторика: Задачи}

\begin{problem}
	Вычислите коэффициент при $ x^{100} $ многочлене $\left(1+x+x^{2}+\ldots+x^{100}\right)^{3}$ после приведения всехподобных членов.
\end{problem}

	
\end{document} % конец документа
