\begin{definition}
	Пусть $R$ — произвольное кольцо. Если существует такое целое положительное число $n$, что $\forall r\in R$ выполняется равенство
	
	$$
	n\cdot r=\underbrace{r+\cdots +r}_{n} = 0
	$$
	то наименьшее из таких чисел n называвается характеристикой поля и обозначается $ {\mathop  {{\mathrm  {char}}}}R$
	Если такого числа не существует то $ {\mathop  {{\mathrm  {char}}}}R = 0$
\end{definition}

\begin{theorem}[\protect{\cite[ВА II \S 3.4 ]{kostrikin2}}]
	\label{thm:thm1}
	\mylabel{text:thm1}{Критерий диагонализируемости}
	Линейный оператор $ A $ с простым спектром диагонализируем
\end{theorem}

\begin{theorem}[\protect{\cite[ВА II \S 3.4 ]{kostrikin2}}]
	\label{thm:thm2}
	\mylabel{text:thm2}{Критерий диагонализируемости}
	Пусть $ A $ линейный оператор на конечномерном векторном пространстве $ V $ на полем $\mathfrak{K}$. Для диагонализируемости $ \mathcal{A} $ необходимо и достаточно выполнение двух условий:
	\begin{itemize}
		\item все корни характеристического многочлена $\chi_{\mathcal{A}}(t)$ лежат в $ \mathfrak{K} $
		\item 
		геометрическая кратность каждого собственного значения  $ \lambda $ совпадает с его алгебраической кратностью.  
	\end{itemize}
\end{theorem}