\documentclass[a4paper,12pt]{article}

%%% Работа с русским языком
\usepackage{cmap}					% поиск в PDF
\usepackage[T2A]{fontenc}			% кодировка
\usepackage[utf8]{inputenc}			% кодировка исходного текста
\usepackage[english,russian]{babel}	% локализация и переносы
%%
\usepackage{cmap} 
\usepackage{ gensymb }
\usepackage[unicode]{hyperref}
\usepackage{ textcomp }
\usepackage{datetime}
\usepackage{physics}
\usepackage{cancel}
\usepackage{mathtools}
\usepackage[margin=0.7in]{geometry}
\usepackage{fancyhdr}
\pagestyle{fancy}
%%

%%% Дополнительная работа с математикой
\usepackage{amsfonts,amssymb,amsthm,mathtools} % AMS
\usepackage{amsmath}
\usepackage{icomma} % "Умная" запятая: $0,2$ --- число, $0, 2$ --- перечисление

%% Теоремы
\newtheorem{exercise}{Упражнение}[section]

\theoremstyle{definition}
\newtheorem{definition}{Определение}
\theoremstyle{plain}
\newtheorem{theorem}{Теорема}
\newtheorem{lemma}{Лемма}
\newtheorem{example}{Пример}
\newtheorem{problem}{Задача}
\newtheorem*{solution}{Решение}

%
\theoremstyle{plain}
%\newtheorem{proof}{Доказательство}
%
\newtheorem{statement}{Утверждение}

\theoremstyle{remark}
\newtheorem*{properties}{Свойства}

%%

\fancyhead[L]{\footnotesize ШАД Яндекс}
\fancyhead[RO]{Последняя компиляция: \today }
\fancyfoot[R]{\thepage}
\fancyfoot[C]{}




%% Номера формул
%\mathtoolsset{showonlyrefs=true} % Показывать номера только у тех формул, на которые есть \eqref{} в тексте.

%% Шрифты
\usepackage{euscript}	 % Шрифт Евклид
\usepackage{mathrsfs} % Красивый матшрифт

%% Свои команды
\DeclareMathOperator{\sgn}{\mathop{sgn}}

%% Перенос знаков в формулах (по Львовскому)
\newcommand*{\hm}[1]{#1\nobreak\discretionary{}
	{\hbox{$\mathsurround=0pt #1$}}{}}

%%% Работа с картинками
\usepackage{graphicx}  % Для вставки рисунков
\graphicspath{{images/}{images2/}}  % папки с картинками
\setlength\fboxsep{3pt} % Отступ рамки \fbox{} от рисунка
\setlength\fboxrule{1pt} % Толщина линий рамки \fbox{}
\usepackage{wrapfig} % Обтекание рисунков и таблиц текстом

%%% Работа с таблицами
\usepackage{array,tabularx,tabulary,booktabs} % Дополнительная работа с таблицами
\usepackage{longtable}  % Длинные таблицы
\usepackage{multirow} % Слияние строк в таблице


%%% Заголовок
\title{Школа анализа данных}
\date{\today}


\begin{document} % конец преамбулы, начало документа
	
	\begin{definition}
	Пусть $R$ — произвольное кольцо. Если существует такое целое положительное число $n$, что $\forall r\in R$ выполняется равенство
	
	$$
	n\cdot r=\underbrace{r+\cdots +r}_{n} = 0
	$$
	то наименьшее из таких чисел n называвается характеристикой поля и обозначается $ {\mathop  {{\mathrm  {char}}}}R$
	Если такого числа не существует то $ {\mathop  {{\mathrm  {char}}}}R = 0$
\end{definition}
	\section{Комбинаторика}
\begin{lemma}[Полиномиальная формула]
	$$\left(x_{1}+x_{2}+\ldots+x_{m}\right)^{n} = \left(x_{1}+x_{2}+\ldots+x_{m}\right)\cdot\left(x_{1}+x_{2}+\ldots+x_{m}\right) \ldots \left(x_{1}+x_{2}+\ldots+x_{m}\right)$$
	
	$$
	k_{1}+k_{2}+\ldots+k_{m}=n
	$$
\end{lemma}

	\section{Линейная Алгебра: Задачи}
\begin{problem}
	Существуют ли матрицы $A$ b $ B $ такие что $A B-B A=I$
\end{problem}

\begin{solution}
	Нет. $\operatorname{tr}(A B)=\operatorname{tr}(B A)$  $\operatorname{tr}(A B-B A)=0$ тогда как
	$\operatorname{tr}(I) \neq 0$
\end{solution}
		
	\section{Теория вероятностей: Задачи}
\begin{problem}{(ФКН ВШЭ: Теория вероятностей: Листок 3)}
	Пусть $ \xi $ некоторая случайная величина. При каком $ a \in \mathbb{R} $ достигается минимальное значение $ f(a) =\mathbb{E}\left[(\xi-a)^{2}\right] $
\end{problem}
\begin{solution}
	Раскроем по линейности матожидание. $ \mathbb{E}\left[(\xi-a)^{2}\right]=\mathbb{E}\left[\xi^{2}\right]-2 a \mathbb{E}[\xi]+a^{2} $.\\
	Прибавим и отнимем $ (\mathbb{E}(\xi))^2 $\\
	
	$f(a)=(a-\mathbb{E}[\xi])^{2}+\mathbb{E}\left[\xi^{2}\right]-(\mathbb{E}[\xi])^{2}=(a-\mathbb{E}[\xi])^{2}+\mathbb{D}[\xi]$
	Так как дисперисия не зависит от параметра $ a $ $ \Rightarrow $ минимум достигается при $ (a-\mathbb{E}[\xi])^{2} = 0 $ т.е $ a =\mathbb{E}[\xi] $
\end{solution}
\begin{problem}{(ФКН ВШЭ: Теория вероятностей: Листок 3)}
	Вычислить $\mathbb{E}[\xi], \mathbb{D}[\xi] u \mathbb{E}\left[3^{\xi}\right]$, если $ \xi $ - это
	\begin{enumerate}
		\item пуассоновская случайная величина с параметром $ \lambda \ge 0 $
		
		\item геометрическая случайная величина с параметром $ p \in (0,1) $
	\end{enumerate}	
\end{problem}

\begin{solution}
	\begin{enumerate}
		\begin{enumerate}
			\item
			
			
			$\mathbb{P}[\xi=k]=\frac{\lambda^{k}}{k !} e^{-\lambda}, k \in \mathbb{N}$ \\
			$\mathbb{E}[\xi]=\sum_{k=1}^{\infty} k \cdot \frac{\lambda^{k}}{k !} e^{-\lambda}=\frac{\lambda}{e^{\lambda}} \sum_{k=1}^{\infty} \frac{\lambda^{k-1}}{(k-1) !}=\frac{\lambda}{e^{\lambda}} \cdot e^{\lambda}=\lambda$ \\
			\item
			$\mathbb{D}[\xi]=\mathbb{E}\left[\xi^{2}\right]-(\mathbb{E}[\xi])^{2}=\sum_{k=1}^{\infty} k^{2} \cdot \frac{\lambda^{k}}{k !} e^{-\lambda}-\lambda^{2}=\frac{1}{e^{\lambda}} \sum_{k=1}^{\infty} \frac{k \lambda^{k}}{(k-1) !}-\lambda^{2}$
			
			$\frac{1}{e^{\lambda}} \sum_{k=0}^{\infty} \frac{(k+1) \lambda^{k+1}}{k !}-\lambda^{2}=\frac{\lambda^{2}}{e^{\lambda}} \sum_{k=1}^{\infty} \frac{\lambda^{k-1}}{(k-1) !}+\frac{\lambda}{e^{\lambda}} \underbrace{\sum_{k=0}^{\infty} \frac{\lambda^{k}}{k !}}_{=e^{\lambda}}-\lambda^{2}=\lambda$
			
			\item
			$\mathbb{E}\left[3^{\xi}\right]=\sum_{k=1}^{\infty} 3^{k} \cdot \frac{\lambda^{k}}{k !} e^{-\lambda}=e^{-\lambda} \underbrace{\sum_{k=1}^{\infty} \frac{(3 \lambda)^{k}}{k !}}_{=e^{3 \lambda}}=e^{2 \lambda}$
		\end{enumerate}
	
	\begin{enumerate}
		\item 
		$\mathbb{P}[\xi=k]=p(1-p)^{k-1}, k \in \mathbb{N}$
		$\mathbb{E}[\xi]=\sum_{k=1}^{\infty} k \cdot(1-p)^{k-1} p=p \sum_{k=1}^{\infty} k(1-p)^{k-1}$
		
		$$
		\sum_{k=1}^{\infty} q^{k}=\frac{q}{1-q} \quad q \in(0,1)
		$$
		
		$$
		\frac{\partial}{\partial q} \sum_{k=1}^{\infty} q^{k}=\sum_{k=1}^{\infty} k q^{k-1}
		$$
		
		$q=1-p$
		
		$$
		\mathbb{E}[\xi]=p\left(\frac{\partial}{\partial q}\left(\frac{q}{1-q}\right) | q=1-p\right)=p\left(\frac{1-q+q}{(1-q)^{2}} | q=1-p\right)=p \cdot \frac{1}{p^{2}}=\frac{1}{p}
		$$
		
		\item 
		$$
		\mathbb{D}[\xi]=\mathbb{E}\left[\xi^{2}\right]-(\mathbb{E}[\xi])^{2}=\mathbb{E}\left[\xi^{2}-\xi+\xi\right]-(\mathbb{E}[\xi])^{2}=\mathbb{E}[\xi(\xi-1)]+\mathbb{E}[\xi]-(\mathbb{E}[\xi])^{2}
		$$
		
		$$
		\begin{aligned} \mathbb{E}[\xi(\xi-1)] = p \sum_{k=1}^{\infty} k(k-1) q^{k-1}=p \cdot \frac{\partial}{\partial q}\left(\sum_{k=1}^{\infty}(k-1) q^{k}\right)=p \cdot \frac{\partial}{\partial q}\left(q^{2} \sum_{k=2}^{\infty}(k-1) q^{k-2}\right)=\\=p \cdot \frac{\partial}{\partial q}\left(q^{2} \cdot \frac{\partial}{\partial q} \sum_{k=2}^{\infty} q^{k-1}\right)=\\\left.=p \cdot \frac{\partial}{\partial q}\left(q^{2} \cdot \frac{\partial}{\partial q} \sum_{k=1}^{\infty} q^{k}\right)=p \cdot \frac{\partial}{\partial q}\left(q^{2} \cdot \frac{\partial}{\partial q}\right)\right)=p \cdot \frac{\partial}{\partial q}\left(\frac{q^{2}}{(1-q)^{2}}\right)=p\left(\frac{-2 q}{(q-1)^{3}}\right)=\\=\{p=1-q\}=p\left(\frac{-2+2 p}{-p^{3}}\right)=\frac{2(p-1)}{-p^{2}}=\frac{2(1-p)}{p^{2}} \end{aligned}
		$$
		
		$$\mathbb{D}[\xi]=\mathbb{E}[\xi(\xi-1)]+\mathbb{E}[\xi]-(\mathbb{E}[\xi])^{2}=\frac{2(1-p)}{p^{2}}+\frac{1}{p}-\frac{1}{p^{2}}=\frac{2-2 p+p-1}{p^{2}}=\frac{1-p}{p^{2}}
		$$
		
		\item 
	
		$$\mathbb{E}\left[3^{\xi}\right]=\sum_{k=1}^{\infty} 3^{k} \cdot p(1-p)^{k-1}=3 p \sum_{k=1}^{\infty} 3^{k-1} \cdot(1-p)^{k-1}$$
		
		Для сходимости необходимо, чтобы $3(1-p)<1$
		
		$$
		\mathbb{E}\left[3^{\xi}\right]=\frac{3 p}{1-3(1-p)}
		$$
	\end{enumerate}

	\end{enumerate}
\end{solution}
	\section{Комбинаторика: Задачи}

\begin{problem}
	Вычислите коэффициент при $ x^{100} $ многочлене $\left(1+x+x^{2}+\ldots+x^{100}\right)^{3}$ после приведения всехподобных членов.
\end{problem}

	
\end{document} % конец документа
