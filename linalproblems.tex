\section{Линейная Алгебра: Задачи}
\begin{problem}
	Существуют ли матрицы $A$ и $ B $ такие что $A B-B A=I$
\end{problem}

\begin{solution}
	Нет. $\operatorname{tr}(A B)=\operatorname{tr}(B A)$  $\operatorname{tr}(A B-B A)=0$ тогда как
	$\operatorname{tr}(I) \neq 0$
\end{solution}

\begin{problem}[ШАД Экзамен 25 мая 2019]
 Матрицы $ A $ и $ B $ таковы, что $ A^2 = A $ и $ B^2 = B $ и матрица $ E - (A + B) $ обратима. Докажите, что $ \operatorname{rk} A = \operatorname{rk} B $

\end{problem}
\begin{solution}
	$ C = E - (A+B) $. Домножим на $ A C =  A \left(E - (A+B)\right) = A - (A^2 + AB) = - AB $. $ CB = \left(E - (A+B)\right) B  = - AB $, Получаем $ AC = CB \rightarrow A = CBC^{-1}$
	
\end{solution}	

\begin{solution}
$\operatorname{rk}(A+B) \leqslant \operatorname{rk}A+\operatorname{rk}B$

$$
\left\{\begin{array}{l}{\operatorname{rk} (E-(A+B))} \leqslant \operatorname{rk}(E-A) + \operatorname{rk} B \\ {\operatorname{rk}(E-(A+B))}  \leqslant \operatorname{rk}(E-B) + \operatorname{rk} A \end{array}\right.
$$


$$
\operatorname{rk}(A) = n_1, \quad \operatorname{rk}(B) = n_2
$$

По свойству проекционных операторов 
$
 \operatorname{rk}(E-A) = n - n_1 , \operatorname{rk}(E-B) = n - n_2
$

$$
\left\{\begin{array}{l}
{n\leqslant n - n_1 + n_2}\\{n\leqslant n - n_2 + n_1}
		
\end{array}\right. \Longrightarrow n_1 = n_2
$$		
\end{solution}

\begin{problem}[ШАД Экзамен 1 июня 2019]
	При каком значении параметра $ a \in \mathbb{R}  $ матрицы
	$$
	A=\left(\begin{array}{cc}{1} & {4-a-a^{2}} \\ {2} & {-1}\end{array}\right) \quad B=\left(\begin{array}{cc}{-a-1} & {3} \\ {3} & {-5}\end{array}\right)
	$$
	могут быть матрицами одной и той же билинейной формы $
	V \times V \rightarrow \mathbb{R}
	$ в различных базисах
	
\end{problem}

\begin{solution}
	$$
	A=C^{T} B C
	\Longrightarrow 
	\operatorname{r k} A=\operatorname{r k} B
	$$
	
	$$
	\left\{\begin{array}{l}{\operatorname{det} A \neq 0} \\ {\operatorname{det} B \neq 0}\end{array}\right.
	$$
	
	$$
	\begin{array}{l}{a \neq \frac{4}{5}} \\ {a \neq \frac{1}{2}(-1-\sqrt{19})} \\ {a \neq \frac{1}{2}(-1+\sqrt{19})}\end{array}
	$$
	
	Заметим, что  $
	B^{T}=B \Longrightarrow 
	A^{T}=A , 
	4-a-a^{2}=2 , a = 1 , a = -2	
	$
\end{solution}


	
	
\begin{problem}[ШАД Экзамен 9 июня 2018]
	Пусть $ A $ и $ B $ -- ортогональные матрицы. Докажите что 
	$ \displaystyle
	 \operatorname{det}\left(A^{\mathrm{T}} B-B^{\mathrm{T}} A\right)=\operatorname{det}(A+B) \cdot \operatorname{det}(A-B)
	$
\end{problem}

\begin{solution}
	$
	\operatorname{det}(A+B)=\operatorname{det}(A+B)^{T}=\operatorname{det}\left(A^{T}+B^{T}\right)
	$
	\\
	$ 
	\operatorname{det}\left(A^{T}+B^{T}\right) \cdot \operatorname{det}(A-B)= \operatorname{det}(\left(A^{ T}+B^{T}\right)(A-B)) = \operatorname{det}\left(A^{T} A-A^{T} B+B^{T} A-B^{T} B\right) = \operatorname{det}\left(E-A^{T} B+B^{T} A-E\right) = 
	\operatorname{det}\left(B^{T} A-A^{T} B\right) = \operatorname{det}\left(B^{T} A-A^{T} B\right)^{T}=
	\operatorname{det}\left(A^{T} B-B^{T} A\right)
	$
\end{solution}	


\begin{problem}[ШАД Экзамен 2 июня 2018]
	Линейный оператор 
$
A: \mathbb{R}^{n} \longrightarrow \mathbb{R}^{n}
$
таков, что это $A^{3}$ - оператор проекции
Какие собственные значения может иметь $A$?  Верно ли, что $A$ будет иметь диагональную матрицу в каком-либо базисе $\mathbb{R}^{n}$?
\end{problem}		

\begin{solution}
	$ A^{3} = P $ \\
	$ P^2 = P $ - проектор,
	$ (A^3)^2 = A^3 $.
	Пусть $ \boldsymbol{v} $ - cобственный вектор матрицы $ A $.
	
	Тогда $ A^6 \boldsymbol{v} = \lambda^6 \boldsymbol{v} = A^{3} \boldsymbol{v} = \lambda^3 \boldsymbol{v}.$
	Так как $ \boldsymbol{v}\neq \boldsymbol{0} $
	$
	\left[\begin{array}{l}{\lambda=0} \\ {\lambda^{3}=1}\end{array}\right.
	$
	
	$
	\begin{array}{l}{\lambda_{1}=e^{\frac{i 2 \pi}{3}}} \\ {\lambda_{2}=e^{\frac{i 4 \pi}{3}}} \\ {\lambda_{3}=e^{i 2 \pi}}\end{array}
	$ - простые корни характеристического многочлена $\chi_{\mathcal{A}}(t)$ , к тому же $ \in \mathbb{C} $
	
	а корень $ \lambda = 0 $ вырожден трёхкратно
	
	Тогда по теореме \ref{text:thm2} следует что оператор $ A $ не диагонализируем
	
\end{solution}

				